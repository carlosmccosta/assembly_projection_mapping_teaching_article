\section{Introduction}\label{sec:introduction}

Teaching humans new manufacturing skills or advanced assembly / maintenance operations can be a long and error prone process that requires reading extensive manuals and a period of tutoring in which they are trained by field experts. This teaching period can be shortened and done without the need of other workers by relying on immersive \gls{hmi} teaching systems that are able to transmit the knowledge more effectively using step by step instructions containing text and video along with visual cues showing the work areas and pick / place locations tagged with contextual help. Moreover, when coupled with active perception systems that can detect the assembly objects and what the operator is doing, the teaching systems can also act as a supervisor, alerting the operator when a mistake is made or when a damaged component needs to be replaced. This approach to skill transfer using immersive \glspl{hmi} along with dynamic feedback can speedup and improve the effectiveness of the training sessions while also giving continuous quality control, allowing to reduce the cost and time of product assembly, repair and maintenance.

With these goals in mind, a spatial augmented reality teaching system was developed for projecting into the operator workspace contextual assembly instructions that provide detailed information about the operations and tools that are required to assemble a given product. The main advantage of this approach is its ability to provide accurate 3D information directly into the environment and only when it is needed. For production lines that may receive custom products or have their employees rotating between workstations, a spatial augmented reality system can quickly and intuitively guide them throughout the assembly process. Moreover, the proposed system can also be used for coordinating tasks between human operators and robotic systems, because it is able to highlight 3D work areas or objects and provide visual cues for informing the operator what the robot will be doing next and where it will be working.

In the following section, a brief overview of the augmented reality systems that were developed over the years will be given. Then in \cref{sec:sar}, the mathematical modeling and calibration of video projectors will be discussed. Later on, \cref{sec:human-machine-interaction} will describe the immersive \gls{hmi} that was developed. Given the lack of \gls{cad} models of the starter motor used for testing our system, \cref{sec:object-reconstruction} will describe how the 3D model was retrieved using a structured light 3D scanner. Then, \cref{sec:pose-estimation} will present the 6 \gls{dof} object pose estimation system. Finally, \cref{sec:training} will discuss the results of a training session while \cref{sec:conclusions} will summarize the conclusions and present possible future work.
