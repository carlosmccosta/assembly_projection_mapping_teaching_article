\section{Conclusions}\label{sec:conclusions}

This paper presented the fundamental technologies required to implement an accurate \gls{sar} system within the domain of small parts assembly. Namely, it proposed an efficient and accurate approach to model video projectors using the \gls{opengl} projection matrix, discussed how to tackle the problem of projecting virtual 3D geometry on top of movable objects using a 6 \gls{dof} pose estimation system and also described how to calibrate the intrinsic and extrinsic parameters of the sensing and projection hardware to ensure that the system as a hole had an overlap error between the physical and virtual objects below 2 mm. These technologies were validated using our immersive teaching system, that is capable of guiding the operator during the assembly process using a projected \gls{hmi} containing text and video content while also providing a visual inspection phase in which the expected product outline is overlaid on top of the assembled components. This prof-of-concept use case served to validate the approaches suggested and can be used as a starting point for other applications, namely cooperative workstations in which a \gls{sar} system can be used to coordinate the tasks between robots and operators by showing directly on the environment what are the expected work areas and operations that are associated to each robot and operator.

The presented immersive teaching system can be improved further by adding an assembly analysis module for monitoring what the operator is doing in order to provide contextual help (such as detecting that the current component was mounted correctly and projecting the next part that the operator needs to assemble) and also alert for possible mistakes done by the operator during assembly. This would allow continuous analysis and quality control of the assembly process, reducing the time required for the detection and correction of assembly problems. On the other hand, the expansion of the \gls{sar} system to other use cases and its evaluation with a large group of operators would provide useful feedback for its improvement and would allow to determine how much effective it can be when compared with traditional methods.
