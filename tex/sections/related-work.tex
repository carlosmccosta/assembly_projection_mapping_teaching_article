\section{Related Work}\label{sec:related-work}

\gls{ar} interfaces \cite{Bimber2005} have a wide range of applications within the manufacturing industry, from the design \cite{Nee2012}, simulation and planning phase for fast prototyping, to the training and guidance \cite{Webel2011} of the operators that will be manufacturing, assembling and providing maintenance for the final product \cite{Zhu2014}. They offer an immersive way of exchanging information between a human operator and a robot / machine, allowing the development of cooperative assembly lines \cite{Lenz2011}. This immersive environment can be created with a wide range of devices, such as projectors, smart glasses, tablets, smart-phones, \gls{vr} headsets, among others. Projectors allow to perform accurate overlay of virtual information into physical objects when properly calibrated \cite{Moreno2012} and modeled within a 3D rendering engine. They are typically installed on top of the workstations for spatially augmenting the environment with digital information without requiring the operator to wear any special hardware (that may cause discomfort or impact their productivity). This approach is very useful for assisting operators in new and complex operations while also helping them perform their tasks faster by having dynamic contextual information shown directly into the environment where it is needed (such as projection of geometric information for assembly / maintenance \cite{Uva2018}, cutting / welding \cite{Doshi2017} or even painting operations \cite{Barbosa2014}).

Wearable devices such as smart glasses / watches offer a flexible alternative \cite{MICHALOS2018194} which is more suitable for providing guiding information when the operator is performing complex jobs in environments that are hard to reach by an overhead projector or when the operator needs to navigate in a large workspace. Screens with rear mounted sensors provide a quick and low cost approach for adding environment annotations which are useful for assembly / maintenance operations. They may use \gls{ar} markers \cite{Siltanen2012}, markerless 2D / 3D perception \cite{Guo2014} or a combination of both to analyze the environment and detect where are the target objects and what the operator is doing \cite{Bannat2008} in order to overlay virtual models on top of the real objects or provide contextual and corrective information to help the operator work faster while also doing less mistakes. The exchange of information between a training system and an operator may take advantage of multimodal \glspl{hmi} \cite{Webel2013} for effectively guiding the operator during the assembly / maintenance tasks (using vibrotactile bracelets along with visual and audio cues). On the other hand, \gls{vr} headsets provide an immersive virtual environment for teaching the robot / operator \cite{ABATE2009318} without requiring access to the physical objects / robots / environment layout, allowing fast testing and prototyping of new products.
