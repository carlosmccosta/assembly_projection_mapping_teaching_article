\section{Related Work}\label{sec:related-work}

\gls{ar} interfaces \cite{Bimber2005} have a wide range of applications within the manufacturing industry \cite{Nee2012,Wang2016}, from the design, simulation \cite{WangOng2016} and planning phase for fast prototyping, to the training \cite{Gorecky2011} and guidance \cite{Webel2011,Gavish2015} of the operators that will be manufacturing, assembling and providing maintenance for the final product. They offer an immersive way of exchanging information between a human operator and a robot / machine \cite{Kollatsch2014,Gaschler2014,Dini2015,Michalos2016}, allowing the development of cooperative assembly lines \cite{Lenz2011,Michalos2014,Makris2016}. This immersive environment can be created with a wide range of devices, such as projectors, smart glasses, tablets, smart-phones, \gls{vr} headsets, among others. Projectors allow to perform accurate environment marking of information \cite{Tan2013,Fujimoto2014}, which is useful for assisting the operator in new complex operations (such as assembly or maintenance) and also allow the operator to perform their tasks faster by having dynamic contextual information either shown alongside the targeted objects or directly into the environment (such as marking of geometry information for cutting / welding operations). Smart glasses (such as the Microsoft HoloLens\footnote{\url{https://www.microsoft.com/microsoft-hololens}}) offer a more flexible alternative which is more suitable for providing guiding information while the operator is performing complex and long jobs. Screens with rear mounted sensors (such as 2D / stereo cameras, RGB-D, \gls{tof}) provide a quick and low cost approach for adding environment annotations which are useful for assembly / maintenance operations. They may use \gls{ar} markers \cite{Siltanen2012}, markerless 2D / 3D perception \cite{Andreopoulos2013,Guo2014} or a combination of both \cite{Wang2009} to analyze the environment and detect where are the target objects are and what the operator is doing \cite{Bannat2008} in order to overlay virtual models on top of the real objects or provide contextual and corrective information to help the operator work faster while also doing less mistakes. This exchange of information may take advantage of multimodal \gls{hmi} \cite{Webel2013} in order to effectively guide the operator (using for example vibrotactile bracelets, along with visual and audio cues) during the assembly / maintenance tasks. On the other hand, \gls{vr} headsets (such as the HTC Vive\footnote{\url{https://www.vive.com}}) provide an immersive virtual environment for teaching the robot / operator without requiring access to the physical objects / robots / environment layout, allowing fast testing and prototyping of new products.
